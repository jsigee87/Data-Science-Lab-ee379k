
% Default to the notebook output style

    


% Inherit from the specified cell style.




    
\documentclass[11pt]{article}

    
    
    \usepackage[T1]{fontenc}
    % Nicer default font (+ math font) than Computer Modern for most use cases
    \usepackage{mathpazo}

    % Basic figure setup, for now with no caption control since it's done
    % automatically by Pandoc (which extracts ![](path) syntax from Markdown).
    \usepackage{graphicx}
    % We will generate all images so they have a width \maxwidth. This means
    % that they will get their normal width if they fit onto the page, but
    % are scaled down if they would overflow the margins.
    \makeatletter
    \def\maxwidth{\ifdim\Gin@nat@width>\linewidth\linewidth
    \else\Gin@nat@width\fi}
    \makeatother
    \let\Oldincludegraphics\includegraphics
    % Set max figure width to be 80% of text width, for now hardcoded.
    \renewcommand{\includegraphics}[1]{\Oldincludegraphics[width=.8\maxwidth]{#1}}
    % Ensure that by default, figures have no caption (until we provide a
    % proper Figure object with a Caption API and a way to capture that
    % in the conversion process - todo).
    \usepackage{caption}
    \DeclareCaptionLabelFormat{nolabel}{}
    \captionsetup{labelformat=nolabel}

    \usepackage{adjustbox} % Used to constrain images to a maximum size 
    \usepackage{xcolor} % Allow colors to be defined
    \usepackage{enumerate} % Needed for markdown enumerations to work
    \usepackage{geometry} % Used to adjust the document margins
    \usepackage{amsmath} % Equations
    \usepackage{amssymb} % Equations
    \usepackage{textcomp} % defines textquotesingle
    % Hack from http://tex.stackexchange.com/a/47451/13684:
    \AtBeginDocument{%
        \def\PYZsq{\textquotesingle}% Upright quotes in Pygmentized code
    }
    \usepackage{upquote} % Upright quotes for verbatim code
    \usepackage{eurosym} % defines \euro
    \usepackage[mathletters]{ucs} % Extended unicode (utf-8) support
    \usepackage[utf8x]{inputenc} % Allow utf-8 characters in the tex document
    \usepackage{fancyvrb} % verbatim replacement that allows latex
    \usepackage{grffile} % extends the file name processing of package graphics 
                         % to support a larger range 
    % The hyperref package gives us a pdf with properly built
    % internal navigation ('pdf bookmarks' for the table of contents,
    % internal cross-reference links, web links for URLs, etc.)
    \usepackage{hyperref}
    \usepackage{longtable} % longtable support required by pandoc >1.10
    \usepackage{booktabs}  % table support for pandoc > 1.12.2
    \usepackage[inline]{enumitem} % IRkernel/repr support (it uses the enumerate* environment)
    \usepackage[normalem]{ulem} % ulem is needed to support strikethroughs (\sout)
                                % normalem makes italics be italics, not underlines
    

    
    
    % Colors for the hyperref package
    \definecolor{urlcolor}{rgb}{0,.145,.698}
    \definecolor{linkcolor}{rgb}{.71,0.21,0.01}
    \definecolor{citecolor}{rgb}{.12,.54,.11}

    % ANSI colors
    \definecolor{ansi-black}{HTML}{3E424D}
    \definecolor{ansi-black-intense}{HTML}{282C36}
    \definecolor{ansi-red}{HTML}{E75C58}
    \definecolor{ansi-red-intense}{HTML}{B22B31}
    \definecolor{ansi-green}{HTML}{00A250}
    \definecolor{ansi-green-intense}{HTML}{007427}
    \definecolor{ansi-yellow}{HTML}{DDB62B}
    \definecolor{ansi-yellow-intense}{HTML}{B27D12}
    \definecolor{ansi-blue}{HTML}{208FFB}
    \definecolor{ansi-blue-intense}{HTML}{0065CA}
    \definecolor{ansi-magenta}{HTML}{D160C4}
    \definecolor{ansi-magenta-intense}{HTML}{A03196}
    \definecolor{ansi-cyan}{HTML}{60C6C8}
    \definecolor{ansi-cyan-intense}{HTML}{258F8F}
    \definecolor{ansi-white}{HTML}{C5C1B4}
    \definecolor{ansi-white-intense}{HTML}{A1A6B2}

    % commands and environments needed by pandoc snippets
    % extracted from the output of `pandoc -s`
    \providecommand{\tightlist}{%
      \setlength{\itemsep}{0pt}\setlength{\parskip}{0pt}}
    \DefineVerbatimEnvironment{Highlighting}{Verbatim}{commandchars=\\\{\}}
    % Add ',fontsize=\small' for more characters per line
    \newenvironment{Shaded}{}{}
    \newcommand{\KeywordTok}[1]{\textcolor[rgb]{0.00,0.44,0.13}{\textbf{{#1}}}}
    \newcommand{\DataTypeTok}[1]{\textcolor[rgb]{0.56,0.13,0.00}{{#1}}}
    \newcommand{\DecValTok}[1]{\textcolor[rgb]{0.25,0.63,0.44}{{#1}}}
    \newcommand{\BaseNTok}[1]{\textcolor[rgb]{0.25,0.63,0.44}{{#1}}}
    \newcommand{\FloatTok}[1]{\textcolor[rgb]{0.25,0.63,0.44}{{#1}}}
    \newcommand{\CharTok}[1]{\textcolor[rgb]{0.25,0.44,0.63}{{#1}}}
    \newcommand{\StringTok}[1]{\textcolor[rgb]{0.25,0.44,0.63}{{#1}}}
    \newcommand{\CommentTok}[1]{\textcolor[rgb]{0.38,0.63,0.69}{\textit{{#1}}}}
    \newcommand{\OtherTok}[1]{\textcolor[rgb]{0.00,0.44,0.13}{{#1}}}
    \newcommand{\AlertTok}[1]{\textcolor[rgb]{1.00,0.00,0.00}{\textbf{{#1}}}}
    \newcommand{\FunctionTok}[1]{\textcolor[rgb]{0.02,0.16,0.49}{{#1}}}
    \newcommand{\RegionMarkerTok}[1]{{#1}}
    \newcommand{\ErrorTok}[1]{\textcolor[rgb]{1.00,0.00,0.00}{\textbf{{#1}}}}
    \newcommand{\NormalTok}[1]{{#1}}
    
    % Additional commands for more recent versions of Pandoc
    \newcommand{\ConstantTok}[1]{\textcolor[rgb]{0.53,0.00,0.00}{{#1}}}
    \newcommand{\SpecialCharTok}[1]{\textcolor[rgb]{0.25,0.44,0.63}{{#1}}}
    \newcommand{\VerbatimStringTok}[1]{\textcolor[rgb]{0.25,0.44,0.63}{{#1}}}
    \newcommand{\SpecialStringTok}[1]{\textcolor[rgb]{0.73,0.40,0.53}{{#1}}}
    \newcommand{\ImportTok}[1]{{#1}}
    \newcommand{\DocumentationTok}[1]{\textcolor[rgb]{0.73,0.13,0.13}{\textit{{#1}}}}
    \newcommand{\AnnotationTok}[1]{\textcolor[rgb]{0.38,0.63,0.69}{\textbf{\textit{{#1}}}}}
    \newcommand{\CommentVarTok}[1]{\textcolor[rgb]{0.38,0.63,0.69}{\textbf{\textit{{#1}}}}}
    \newcommand{\VariableTok}[1]{\textcolor[rgb]{0.10,0.09,0.49}{{#1}}}
    \newcommand{\ControlFlowTok}[1]{\textcolor[rgb]{0.00,0.44,0.13}{\textbf{{#1}}}}
    \newcommand{\OperatorTok}[1]{\textcolor[rgb]{0.40,0.40,0.40}{{#1}}}
    \newcommand{\BuiltInTok}[1]{{#1}}
    \newcommand{\ExtensionTok}[1]{{#1}}
    \newcommand{\PreprocessorTok}[1]{\textcolor[rgb]{0.74,0.48,0.00}{{#1}}}
    \newcommand{\AttributeTok}[1]{\textcolor[rgb]{0.49,0.56,0.16}{{#1}}}
    \newcommand{\InformationTok}[1]{\textcolor[rgb]{0.38,0.63,0.69}{\textbf{\textit{{#1}}}}}
    \newcommand{\WarningTok}[1]{\textcolor[rgb]{0.38,0.63,0.69}{\textbf{\textit{{#1}}}}}
    
    
    % Define a nice break command that doesn't care if a line doesn't already
    % exist.
    \def\br{\hspace*{\fill} \\* }
    % Math Jax compatability definitions
    \def\gt{>}
    \def\lt{<}
    % Document parameters
    \title{lab1\_compiled}
    
    
    

    % Pygments definitions
    
\makeatletter
\def\PY@reset{\let\PY@it=\relax \let\PY@bf=\relax%
    \let\PY@ul=\relax \let\PY@tc=\relax%
    \let\PY@bc=\relax \let\PY@ff=\relax}
\def\PY@tok#1{\csname PY@tok@#1\endcsname}
\def\PY@toks#1+{\ifx\relax#1\empty\else%
    \PY@tok{#1}\expandafter\PY@toks\fi}
\def\PY@do#1{\PY@bc{\PY@tc{\PY@ul{%
    \PY@it{\PY@bf{\PY@ff{#1}}}}}}}
\def\PY#1#2{\PY@reset\PY@toks#1+\relax+\PY@do{#2}}

\expandafter\def\csname PY@tok@w\endcsname{\def\PY@tc##1{\textcolor[rgb]{0.73,0.73,0.73}{##1}}}
\expandafter\def\csname PY@tok@fm\endcsname{\def\PY@tc##1{\textcolor[rgb]{0.00,0.00,1.00}{##1}}}
\expandafter\def\csname PY@tok@kt\endcsname{\def\PY@tc##1{\textcolor[rgb]{0.69,0.00,0.25}{##1}}}
\expandafter\def\csname PY@tok@kr\endcsname{\let\PY@bf=\textbf\def\PY@tc##1{\textcolor[rgb]{0.00,0.50,0.00}{##1}}}
\expandafter\def\csname PY@tok@mb\endcsname{\def\PY@tc##1{\textcolor[rgb]{0.40,0.40,0.40}{##1}}}
\expandafter\def\csname PY@tok@s1\endcsname{\def\PY@tc##1{\textcolor[rgb]{0.73,0.13,0.13}{##1}}}
\expandafter\def\csname PY@tok@cpf\endcsname{\let\PY@it=\textit\def\PY@tc##1{\textcolor[rgb]{0.25,0.50,0.50}{##1}}}
\expandafter\def\csname PY@tok@no\endcsname{\def\PY@tc##1{\textcolor[rgb]{0.53,0.00,0.00}{##1}}}
\expandafter\def\csname PY@tok@sr\endcsname{\def\PY@tc##1{\textcolor[rgb]{0.73,0.40,0.53}{##1}}}
\expandafter\def\csname PY@tok@ge\endcsname{\let\PY@it=\textit}
\expandafter\def\csname PY@tok@gu\endcsname{\let\PY@bf=\textbf\def\PY@tc##1{\textcolor[rgb]{0.50,0.00,0.50}{##1}}}
\expandafter\def\csname PY@tok@s\endcsname{\def\PY@tc##1{\textcolor[rgb]{0.73,0.13,0.13}{##1}}}
\expandafter\def\csname PY@tok@m\endcsname{\def\PY@tc##1{\textcolor[rgb]{0.40,0.40,0.40}{##1}}}
\expandafter\def\csname PY@tok@nc\endcsname{\let\PY@bf=\textbf\def\PY@tc##1{\textcolor[rgb]{0.00,0.00,1.00}{##1}}}
\expandafter\def\csname PY@tok@vi\endcsname{\def\PY@tc##1{\textcolor[rgb]{0.10,0.09,0.49}{##1}}}
\expandafter\def\csname PY@tok@kn\endcsname{\let\PY@bf=\textbf\def\PY@tc##1{\textcolor[rgb]{0.00,0.50,0.00}{##1}}}
\expandafter\def\csname PY@tok@k\endcsname{\let\PY@bf=\textbf\def\PY@tc##1{\textcolor[rgb]{0.00,0.50,0.00}{##1}}}
\expandafter\def\csname PY@tok@gr\endcsname{\def\PY@tc##1{\textcolor[rgb]{1.00,0.00,0.00}{##1}}}
\expandafter\def\csname PY@tok@mh\endcsname{\def\PY@tc##1{\textcolor[rgb]{0.40,0.40,0.40}{##1}}}
\expandafter\def\csname PY@tok@na\endcsname{\def\PY@tc##1{\textcolor[rgb]{0.49,0.56,0.16}{##1}}}
\expandafter\def\csname PY@tok@sd\endcsname{\let\PY@it=\textit\def\PY@tc##1{\textcolor[rgb]{0.73,0.13,0.13}{##1}}}
\expandafter\def\csname PY@tok@nd\endcsname{\def\PY@tc##1{\textcolor[rgb]{0.67,0.13,1.00}{##1}}}
\expandafter\def\csname PY@tok@go\endcsname{\def\PY@tc##1{\textcolor[rgb]{0.53,0.53,0.53}{##1}}}
\expandafter\def\csname PY@tok@il\endcsname{\def\PY@tc##1{\textcolor[rgb]{0.40,0.40,0.40}{##1}}}
\expandafter\def\csname PY@tok@vg\endcsname{\def\PY@tc##1{\textcolor[rgb]{0.10,0.09,0.49}{##1}}}
\expandafter\def\csname PY@tok@ne\endcsname{\let\PY@bf=\textbf\def\PY@tc##1{\textcolor[rgb]{0.82,0.25,0.23}{##1}}}
\expandafter\def\csname PY@tok@cm\endcsname{\let\PY@it=\textit\def\PY@tc##1{\textcolor[rgb]{0.25,0.50,0.50}{##1}}}
\expandafter\def\csname PY@tok@sa\endcsname{\def\PY@tc##1{\textcolor[rgb]{0.73,0.13,0.13}{##1}}}
\expandafter\def\csname PY@tok@nb\endcsname{\def\PY@tc##1{\textcolor[rgb]{0.00,0.50,0.00}{##1}}}
\expandafter\def\csname PY@tok@nl\endcsname{\def\PY@tc##1{\textcolor[rgb]{0.63,0.63,0.00}{##1}}}
\expandafter\def\csname PY@tok@vm\endcsname{\def\PY@tc##1{\textcolor[rgb]{0.10,0.09,0.49}{##1}}}
\expandafter\def\csname PY@tok@gt\endcsname{\def\PY@tc##1{\textcolor[rgb]{0.00,0.27,0.87}{##1}}}
\expandafter\def\csname PY@tok@ch\endcsname{\let\PY@it=\textit\def\PY@tc##1{\textcolor[rgb]{0.25,0.50,0.50}{##1}}}
\expandafter\def\csname PY@tok@sc\endcsname{\def\PY@tc##1{\textcolor[rgb]{0.73,0.13,0.13}{##1}}}
\expandafter\def\csname PY@tok@nf\endcsname{\def\PY@tc##1{\textcolor[rgb]{0.00,0.00,1.00}{##1}}}
\expandafter\def\csname PY@tok@kc\endcsname{\let\PY@bf=\textbf\def\PY@tc##1{\textcolor[rgb]{0.00,0.50,0.00}{##1}}}
\expandafter\def\csname PY@tok@vc\endcsname{\def\PY@tc##1{\textcolor[rgb]{0.10,0.09,0.49}{##1}}}
\expandafter\def\csname PY@tok@nn\endcsname{\let\PY@bf=\textbf\def\PY@tc##1{\textcolor[rgb]{0.00,0.00,1.00}{##1}}}
\expandafter\def\csname PY@tok@mo\endcsname{\def\PY@tc##1{\textcolor[rgb]{0.40,0.40,0.40}{##1}}}
\expandafter\def\csname PY@tok@ow\endcsname{\let\PY@bf=\textbf\def\PY@tc##1{\textcolor[rgb]{0.67,0.13,1.00}{##1}}}
\expandafter\def\csname PY@tok@kd\endcsname{\let\PY@bf=\textbf\def\PY@tc##1{\textcolor[rgb]{0.00,0.50,0.00}{##1}}}
\expandafter\def\csname PY@tok@sx\endcsname{\def\PY@tc##1{\textcolor[rgb]{0.00,0.50,0.00}{##1}}}
\expandafter\def\csname PY@tok@dl\endcsname{\def\PY@tc##1{\textcolor[rgb]{0.73,0.13,0.13}{##1}}}
\expandafter\def\csname PY@tok@cp\endcsname{\def\PY@tc##1{\textcolor[rgb]{0.74,0.48,0.00}{##1}}}
\expandafter\def\csname PY@tok@mi\endcsname{\def\PY@tc##1{\textcolor[rgb]{0.40,0.40,0.40}{##1}}}
\expandafter\def\csname PY@tok@se\endcsname{\let\PY@bf=\textbf\def\PY@tc##1{\textcolor[rgb]{0.73,0.40,0.13}{##1}}}
\expandafter\def\csname PY@tok@gs\endcsname{\let\PY@bf=\textbf}
\expandafter\def\csname PY@tok@gp\endcsname{\let\PY@bf=\textbf\def\PY@tc##1{\textcolor[rgb]{0.00,0.00,0.50}{##1}}}
\expandafter\def\csname PY@tok@bp\endcsname{\def\PY@tc##1{\textcolor[rgb]{0.00,0.50,0.00}{##1}}}
\expandafter\def\csname PY@tok@gd\endcsname{\def\PY@tc##1{\textcolor[rgb]{0.63,0.00,0.00}{##1}}}
\expandafter\def\csname PY@tok@c1\endcsname{\let\PY@it=\textit\def\PY@tc##1{\textcolor[rgb]{0.25,0.50,0.50}{##1}}}
\expandafter\def\csname PY@tok@gh\endcsname{\let\PY@bf=\textbf\def\PY@tc##1{\textcolor[rgb]{0.00,0.00,0.50}{##1}}}
\expandafter\def\csname PY@tok@ni\endcsname{\let\PY@bf=\textbf\def\PY@tc##1{\textcolor[rgb]{0.60,0.60,0.60}{##1}}}
\expandafter\def\csname PY@tok@sh\endcsname{\def\PY@tc##1{\textcolor[rgb]{0.73,0.13,0.13}{##1}}}
\expandafter\def\csname PY@tok@kp\endcsname{\def\PY@tc##1{\textcolor[rgb]{0.00,0.50,0.00}{##1}}}
\expandafter\def\csname PY@tok@nt\endcsname{\let\PY@bf=\textbf\def\PY@tc##1{\textcolor[rgb]{0.00,0.50,0.00}{##1}}}
\expandafter\def\csname PY@tok@si\endcsname{\let\PY@bf=\textbf\def\PY@tc##1{\textcolor[rgb]{0.73,0.40,0.53}{##1}}}
\expandafter\def\csname PY@tok@err\endcsname{\def\PY@bc##1{\setlength{\fboxsep}{0pt}\fcolorbox[rgb]{1.00,0.00,0.00}{1,1,1}{\strut ##1}}}
\expandafter\def\csname PY@tok@cs\endcsname{\let\PY@it=\textit\def\PY@tc##1{\textcolor[rgb]{0.25,0.50,0.50}{##1}}}
\expandafter\def\csname PY@tok@gi\endcsname{\def\PY@tc##1{\textcolor[rgb]{0.00,0.63,0.00}{##1}}}
\expandafter\def\csname PY@tok@nv\endcsname{\def\PY@tc##1{\textcolor[rgb]{0.10,0.09,0.49}{##1}}}
\expandafter\def\csname PY@tok@o\endcsname{\def\PY@tc##1{\textcolor[rgb]{0.40,0.40,0.40}{##1}}}
\expandafter\def\csname PY@tok@s2\endcsname{\def\PY@tc##1{\textcolor[rgb]{0.73,0.13,0.13}{##1}}}
\expandafter\def\csname PY@tok@c\endcsname{\let\PY@it=\textit\def\PY@tc##1{\textcolor[rgb]{0.25,0.50,0.50}{##1}}}
\expandafter\def\csname PY@tok@ss\endcsname{\def\PY@tc##1{\textcolor[rgb]{0.10,0.09,0.49}{##1}}}
\expandafter\def\csname PY@tok@sb\endcsname{\def\PY@tc##1{\textcolor[rgb]{0.73,0.13,0.13}{##1}}}
\expandafter\def\csname PY@tok@mf\endcsname{\def\PY@tc##1{\textcolor[rgb]{0.40,0.40,0.40}{##1}}}

\def\PYZbs{\char`\\}
\def\PYZus{\char`\_}
\def\PYZob{\char`\{}
\def\PYZcb{\char`\}}
\def\PYZca{\char`\^}
\def\PYZam{\char`\&}
\def\PYZlt{\char`\<}
\def\PYZgt{\char`\>}
\def\PYZsh{\char`\#}
\def\PYZpc{\char`\%}
\def\PYZdl{\char`\$}
\def\PYZhy{\char`\-}
\def\PYZsq{\char`\'}
\def\PYZdq{\char`\"}
\def\PYZti{\char`\~}
% for compatibility with earlier versions
\def\PYZat{@}
\def\PYZlb{[}
\def\PYZrb{]}
\makeatother


    % Exact colors from NB
    \definecolor{incolor}{rgb}{0.0, 0.0, 0.5}
    \definecolor{outcolor}{rgb}{0.545, 0.0, 0.0}



    
    % Prevent overflowing lines due to hard-to-break entities
    \sloppy 
    % Setup hyperref package
    \hypersetup{
      breaklinks=true,  % so long urls are correctly broken across lines
      colorlinks=true,
      urlcolor=urlcolor,
      linkcolor=linkcolor,
      citecolor=citecolor,
      }
    % Slightly bigger margins than the latex defaults
    
    \geometry{verbose,tmargin=1in,bmargin=1in,lmargin=1in,rmargin=1in}
    
    

    \begin{document}
    
    
    \maketitle
    
    

    
    \begin{Verbatim}[commandchars=\\\{\}]
{\color{incolor}In [{\color{incolor}1}]:} \PY{k+kn}{import} \PY{n+nn}{numpy} \PY{k}{as} \PY{n+nn}{np}
        \PY{k+kn}{import} \PY{n+nn}{matplotlib}\PY{n+nn}{.}\PY{n+nn}{pyplot} \PY{k}{as} \PY{n+nn}{plt}
        \PY{k+kn}{import} \PY{n+nn}{pandas} \PY{k}{as} \PY{n+nn}{pd}
\end{Verbatim}


    \section{Problem 1}\label{problem-1}

Generate two gaussian distributions with mean of 10 and -10, stddev = 5
and size =1000 Add both of the distributions and plot a histogram
Estimate mean and variance

    \begin{Verbatim}[commandchars=\\\{\}]
{\color{incolor}In [{\color{incolor}23}]:} \PY{o}{\PYZpc{}}\PY{k}{matplotlib} inline
         
         \PY{c+c1}{\PYZsh{} Generating two gaussian distributions using numpy function}
         \PY{n}{distribution\PYZus{}1} \PY{o}{=} \PY{n}{np}\PY{o}{.}\PY{n}{random}\PY{o}{.}\PY{n}{normal}\PY{p}{(}\PY{n}{loc} \PY{o}{=} \PY{o}{\PYZhy{}}\PY{l+m+mi}{10} \PY{p}{,} \PY{n}{scale} \PY{o}{=} \PY{l+m+mi}{5} \PY{p}{,} \PY{n}{size} \PY{o}{=} \PY{l+m+mi}{1000}\PY{p}{)}
         \PY{n}{distribution\PYZus{}2} \PY{o}{=} \PY{n}{np}\PY{o}{.}\PY{n}{random}\PY{o}{.}\PY{n}{normal}\PY{p}{(}\PY{n}{loc} \PY{o}{=} \PY{l+m+mi}{10}\PY{p}{,} \PY{n}{scale} \PY{o}{=} \PY{l+m+mi}{5}\PY{p}{,} \PY{n}{size} \PY{o}{=}\PY{l+m+mi}{1000}\PY{p}{)}
         
         \PY{c+c1}{\PYZsh{} Calculating sum of the entries from both the distributions}
         \PY{n}{sum\PYZus{}dist} \PY{o}{=} \PY{n}{distribution\PYZus{}1} \PY{o}{+} \PY{n}{distribution\PYZus{}2}
         
         \PY{c+c1}{\PYZsh{}calculating mean using numpy function mean}
         \PY{n}{mean} \PY{o}{=} \PY{n}{np}\PY{o}{.}\PY{n}{mean}\PY{p}{(}\PY{n}{sum\PYZus{}dist}\PY{p}{)}
         
         \PY{c+c1}{\PYZsh{}calculating variance using numpy function var}
         \PY{n}{var} \PY{o}{=} \PY{n}{np}\PY{o}{.}\PY{n}{var}\PY{p}{(}\PY{n}{sum\PYZus{}dist}\PY{p}{)}
         \PY{n+nb}{print}\PY{p}{(}\PY{l+s+s2}{\PYZdq{}}\PY{l+s+s2}{The estimated mean of the sum of the two generated gaussian distribution is }\PY{l+s+si}{\PYZob{}\PYZcb{}}\PY{l+s+s2}{\PYZdq{}}\PY{o}{.}\PY{n}{format}\PY{p}{(}\PY{n}{mean}\PY{p}{)}\PY{p}{)}
         \PY{n+nb}{print}\PY{p}{(}\PY{l+s+s2}{\PYZdq{}}\PY{l+s+s2}{The estivated variance of the sum of the two generated gaussian distribution is }\PY{l+s+si}{\PYZob{}\PYZcb{}}\PY{l+s+s2}{\PYZdq{}}\PY{o}{.}\PY{n}{format}\PY{p}{(}\PY{n}{var}\PY{p}{)}\PY{p}{)}
         
         \PY{c+c1}{\PYZsh{}plotting histogram using matplotlib of the distribution generated by adding two gaussian distributions}
         \PY{n}{plt}\PY{o}{.}\PY{n}{hist}\PY{p}{(}\PY{n}{sum\PYZus{}dist}\PY{p}{)}
         
         \PY{c+c1}{\PYZsh{}setting x and y labels}
         \PY{n}{plt}\PY{o}{.}\PY{n}{xlabel}\PY{p}{(}\PY{l+s+s2}{\PYZdq{}}\PY{l+s+s2}{Value}\PY{l+s+s2}{\PYZdq{}}\PY{p}{)}
         \PY{n}{plt}\PY{o}{.}\PY{n}{ylabel}\PY{p}{(}\PY{l+s+s2}{\PYZdq{}}\PY{l+s+s2}{Frequency}\PY{l+s+s2}{\PYZdq{}}\PY{p}{)}
         
         \PY{c+c1}{\PYZsh{}setting the title of the plot}
         \PY{n}{plt}\PY{o}{.}\PY{n}{title}\PY{p}{(}\PY{l+s+s2}{\PYZdq{}}\PY{l+s+s2}{Histogram of the sum of two Gaussian distributions}\PY{l+s+s2}{\PYZdq{}}\PY{p}{)}
         
         \PY{c+c1}{\PYZsh{}displaying the plot}
         \PY{n}{plt}\PY{o}{.}\PY{n}{show}\PY{p}{(}\PY{p}{)}
\end{Verbatim}


    \begin{Verbatim}[commandchars=\\\{\}]
The estimated mean of the sum of the two generated gaussian distribution is 0.013391783296487575
The estivated variance of the sum of the two generated gaussian distribution is 51.96203602523229

    \end{Verbatim}

    \begin{center}
    \adjustimage{max size={0.9\linewidth}{0.9\paperheight}}{output_2_1.png}
    \end{center}
    { \hspace*{\fill} \\}
    
    \section{Problem 2}\label{problem-2}

Let Xi be an iid Bernoulli random variable with value f-1,1g. Look at
the random variable Zn = 1/n sum Xi. By taking 1000 draws from Zn, plot
its histogram. Check that for small n (say, 5-10) Zn does not look that
much like a Gaussian, but when n is bigger (already by the time n = 30
or 50) it looks much more like a Gaussian. Check also for much bigger n:
n = 250, to see that at this point, one can really see the bell curve.

    \begin{Verbatim}[commandchars=\\\{\}]
{\color{incolor}In [{\color{incolor}3}]:} \PY{c+c1}{\PYZsh{} Function to generate samples}
        \PY{k}{def} \PY{n+nf}{getSample}\PY{p}{(}\PY{n}{n}\PY{p}{,} \PY{n}{p} \PY{o}{=} \PY{l+m+mf}{0.5}\PY{p}{)}\PY{p}{:}       
            \PY{n}{draws} \PY{o}{=} \PY{n}{np}\PY{o}{.}\PY{n}{ones}\PY{p}{(}\PY{p}{(}\PY{n}{n}\PY{p}{)}\PY{p}{)}
            \PY{k}{for} \PY{n}{i} \PY{o+ow}{in} \PY{n+nb}{range}\PY{p}{(}\PY{n}{n}\PY{p}{)}\PY{p}{:}
                \PY{n}{draw} \PY{o}{=} \PY{n}{np}\PY{o}{.}\PY{n}{random}\PY{o}{.}\PY{n}{uniform}\PY{p}{(}\PY{p}{)}
                \PY{k}{if} \PY{n}{draw} \PY{o}{\PYZlt{}} \PY{n}{p}\PY{p}{:}
                    \PY{n}{draws}\PY{p}{[}\PY{n}{i}\PY{p}{]} \PY{o}{=} \PY{o}{\PYZhy{}}\PY{l+m+mi}{1}
        
            \PY{k}{return} \PY{n}{np}\PY{o}{.}\PY{n}{sum}\PY{p}{(}\PY{n}{draws}\PY{p}{)}\PY{o}{/}\PY{n}{n}
        
        \PY{c+c1}{\PYZsh{} Function to generate historgram}
        \PY{c+c1}{\PYZsh{} takes n as input for number of samples}
        \PY{k}{def} \PY{n+nf}{getHistogram}\PY{p}{(}\PY{n}{n}\PY{p}{,} \PY{n}{num\PYZus{}draws}\PY{o}{=}\PY{l+m+mi}{1000}\PY{p}{)}\PY{p}{:}
            \PY{n}{Zn} \PY{o}{=} \PY{n}{np}\PY{o}{.}\PY{n}{zeros}\PY{p}{(}\PY{p}{(}\PY{n}{num\PYZus{}draws}\PY{p}{)}\PY{p}{)}
            
            \PY{k}{for} \PY{n}{i} \PY{o+ow}{in} \PY{n+nb}{range}\PY{p}{(}\PY{n}{num\PYZus{}draws}\PY{p}{)}\PY{p}{:}
                \PY{n}{Zn}\PY{p}{[}\PY{n}{i}\PY{p}{]} \PY{o}{=} \PY{n}{getSample}\PY{p}{(}\PY{n}{n}\PY{p}{)}
        
            \PY{n}{plt}\PY{o}{.}\PY{n}{title}\PY{p}{(}\PY{l+s+s1}{\PYZsq{}}\PY{l+s+s1}{Histogram of Bernoulli Sample}\PY{l+s+s1}{\PYZsq{}}\PY{p}{)}
            \PY{n}{plt}\PY{o}{.}\PY{n}{xlabel}\PY{p}{(}\PY{l+s+s1}{\PYZsq{}}\PY{l+s+s1}{Sum of }\PY{l+s+si}{\PYZob{}\PYZcb{}}\PY{l+s+s1}{ Bernoulli r.v.s taken }\PY{l+s+si}{\PYZob{}\PYZcb{}}\PY{l+s+s1}{ times}\PY{l+s+s1}{\PYZsq{}}\PY{o}{.}\PY{n}{format}\PY{p}{(}\PY{n}{n}\PY{p}{,} \PY{n}{num\PYZus{}draws}\PY{p}{)}\PY{p}{)}
            \PY{n}{plt}\PY{o}{.}\PY{n}{ylabel}\PY{p}{(}\PY{l+s+s1}{\PYZsq{}}\PY{l+s+s1}{Frequency}\PY{l+s+s1}{\PYZsq{}}\PY{p}{)}
            \PY{n}{plt}\PY{o}{.}\PY{n}{hist}\PY{p}{(}\PY{n}{Zn}\PY{p}{,} \PY{n}{bins} \PY{o}{=} \PY{n}{n}\PY{p}{)}
            \PY{n}{plt}\PY{o}{.}\PY{n}{show}\PY{p}{(}\PY{p}{)}
        
        \PY{l+s+sd}{\PYZsq{}\PYZsq{}\PYZsq{}}
        \PY{l+s+sd}{class Bernoulli:}
        \PY{l+s+sd}{    def \PYZus{}\PYZus{}init\PYZus{}\PYZus{}(self, p):}
        \PY{l+s+sd}{        draw = np.random.uniform()}
        \PY{l+s+sd}{        if draw \PYZgt{} p:}
        \PY{l+s+sd}{            self.value = 1}
        \PY{l+s+sd}{        else:}
        \PY{l+s+sd}{            self.value = \PYZhy{}1}
        \PY{l+s+sd}{\PYZsq{}\PYZsq{}\PYZsq{}}
\end{Verbatim}


\begin{Verbatim}[commandchars=\\\{\}]
{\color{outcolor}Out[{\color{outcolor}3}]:} '\textbackslash{}nclass Bernoulli:\textbackslash{}n    def \_\_init\_\_(self, p):\textbackslash{}n        draw = np.random.uniform()\textbackslash{}n        if draw > p:\textbackslash{}n            self.value = 1\textbackslash{}n        else:\textbackslash{}n            self.value = -1\textbackslash{}n'
\end{Verbatim}
            
    \begin{Verbatim}[commandchars=\\\{\}]
{\color{incolor}In [{\color{incolor}4}]:} \PY{c+c1}{\PYZsh{} Calling getHistogram function for different values of n to notice how the curve }
        \PY{c+c1}{\PYZsh{} changes into bell curve for large n}
        \PY{n}{getHistogram}\PY{p}{(}\PY{l+m+mi}{5}\PY{p}{)}
        \PY{n}{getHistogram}\PY{p}{(}\PY{l+m+mi}{30}\PY{p}{)}
        \PY{n}{getHistogram}\PY{p}{(}\PY{l+m+mi}{50}\PY{p}{)}
        \PY{n}{getHistogram}\PY{p}{(}\PY{l+m+mi}{250}\PY{p}{)}
\end{Verbatim}


    \begin{center}
    \adjustimage{max size={0.9\linewidth}{0.9\paperheight}}{output_5_0.png}
    \end{center}
    { \hspace*{\fill} \\}
    
    \begin{center}
    \adjustimage{max size={0.9\linewidth}{0.9\paperheight}}{output_5_1.png}
    \end{center}
    { \hspace*{\fill} \\}
    
    \begin{center}
    \adjustimage{max size={0.9\linewidth}{0.9\paperheight}}{output_5_2.png}
    \end{center}
    { \hspace*{\fill} \\}
    
    \begin{center}
    \adjustimage{max size={0.9\linewidth}{0.9\paperheight}}{output_5_3.png}
    \end{center}
    { \hspace*{\fill} \\}
    
    \section{Problem 3}\label{problem-3}

Generate 25000 sample points from Gaussian distribution with mean =0,
std dev = 5 Calculate mean and std dev without using library functions

    \begin{Verbatim}[commandchars=\\\{\}]
{\color{incolor}In [{\color{incolor}5}]:} \PY{c+c1}{\PYZsh{} Generating the desired normal distribution with mean = 0, std dev = 5 and 25000 entries}
        \PY{n}{distribution} \PY{o}{=} \PY{n}{np}\PY{o}{.}\PY{n}{random}\PY{o}{.}\PY{n}{normal}\PY{p}{(}\PY{n}{loc} \PY{o}{=}\PY{l+m+mi}{0}\PY{p}{,}\PY{n}{scale}\PY{o}{=}\PY{l+m+mi}{5}\PY{p}{,}\PY{n}{size} \PY{o}{=} \PY{l+m+mi}{25000}\PY{p}{)}
        
        \PY{c+c1}{\PYZsh{} Calculating the mean of generated distribution by adding all the entries and dividing by the number of entries in the }
        \PY{c+c1}{\PYZsh{} distribution}
        \PY{n}{calc\PYZus{}mean} \PY{o}{=} \PY{n}{np}\PY{o}{.}\PY{n}{sum}\PY{p}{(}\PY{n}{distribution}\PY{p}{)}\PY{o}{/}\PY{p}{(}\PY{n}{distribution}\PY{o}{.}\PY{n}{size}\PY{p}{)}
        \PY{n+nb}{print}\PY{p}{(}\PY{l+s+s2}{\PYZdq{}}\PY{l+s+s2}{Mean of sample points from Gaussian is }\PY{l+s+si}{\PYZob{}\PYZcb{}}\PY{l+s+s2}{ }\PY{l+s+s2}{\PYZdq{}}\PY{o}{.}\PY{n}{format}\PY{p}{(}\PY{n}{calc\PYZus{}mean}\PY{p}{)}\PY{p}{)}
        
        \PY{c+c1}{\PYZsh{} Calculating the quantity (X\PYZhy{}E[x])\PYZca{}2 to find variance}
        \PY{n}{distribution\PYZus{}subtracted\PYZus{}mean\PYZus{}square} \PY{o}{=} \PY{p}{(}\PY{n}{distribution} \PY{o}{\PYZhy{}} \PY{n}{calc\PYZus{}mean}\PY{p}{)}\PY{o}{*}\PY{p}{(}\PY{n}{distribution}\PY{o}{\PYZhy{}}\PY{n}{calc\PYZus{}mean}\PY{p}{)}
        
        \PY{c+c1}{\PYZsh{} Calculating variance by using the definition of variance as expected value of (X\PYZhy{}E[x])\PYZca{}2}
        \PY{n}{variance\PYZus{}of\PYZus{}distribution} \PY{o}{=} \PY{n}{np}\PY{o}{.}\PY{n}{sum}\PY{p}{(}\PY{n}{distribution\PYZus{}subtracted\PYZus{}mean\PYZus{}square}\PY{p}{)}\PY{o}{/}\PY{n}{distribution\PYZus{}subtracted\PYZus{}mean\PYZus{}square}\PY{o}{.}\PY{n}{size}
        
        \PY{c+c1}{\PYZsh{} Using the fact that standard deviation is square root of variance to calculate standard deviation}
        \PY{n}{std\PYZus{}dev} \PY{o}{=} \PY{n}{np}\PY{o}{.}\PY{n}{sqrt}\PY{p}{(}\PY{n}{variance\PYZus{}of\PYZus{}distribution}\PY{p}{)}
        
        \PY{n+nb}{print}\PY{p}{(}\PY{l+s+s2}{\PYZdq{}}\PY{l+s+s2}{Standard deviation of the samples points is }\PY{l+s+si}{\PYZob{}\PYZcb{}}\PY{l+s+s2}{\PYZdq{}}\PY{o}{.}\PY{n}{format}\PY{p}{(}\PY{n}{std\PYZus{}dev}\PY{p}{)}\PY{p}{)}
\end{Verbatim}


    \begin{Verbatim}[commandchars=\\\{\}]
Mean of sample points from Gaussian is 0.05107647452877015 
Standard deviation of the samples points is 5.033393089259453

    \end{Verbatim}

    \section{Problem 4}\label{problem-4}

Estimate the mean and covariance matrix for multi-dimensional data:
generate 10,000 samples of 2 dimensional data from the Gaussian
distribution given.

Then, estimate the mean and covariance matrix for this multi-dimensional
data using elemen- tary numpy commands, i.e., addition, multiplication,
division (do not use a command that takes data and returns the mean or
standard deviation).

    \begin{Verbatim}[commandchars=\\\{\}]
{\color{incolor}In [{\color{incolor}6}]:} \PY{n}{mean} \PY{o}{=} \PY{p}{[}\PY{o}{\PYZhy{}}\PY{l+m+mi}{5}\PY{p}{,}\PY{l+m+mi}{5}\PY{p}{]}
        \PY{n}{covariance} \PY{o}{=} \PY{p}{[}\PY{p}{[}\PY{l+m+mi}{20}\PY{p}{,}\PY{l+m+mf}{0.8}\PY{p}{]}\PY{p}{,}\PY{p}{[}\PY{l+m+mf}{0.8}\PY{p}{,}\PY{l+m+mi}{30}\PY{p}{]}\PY{p}{]}
        
        \PY{c+c1}{\PYZsh{}Generating a bivariate gaussian distribution with given mean and covariance matrix}
        \PY{n}{x1}\PY{p}{,}\PY{n}{x2} \PY{o}{=}\PY{n}{np}\PY{o}{.}\PY{n}{random}\PY{o}{.}\PY{n}{multivariate\PYZus{}normal}\PY{p}{(}\PY{n}{mean}\PY{p}{,}\PY{n}{covariance}\PY{p}{,}\PY{l+m+mi}{10000}\PY{p}{)}\PY{o}{.}\PY{n}{T}
        
        \PY{c+c1}{\PYZsh{}Mean of x1 and x2 are simply sum of the values divided by the size of the distribution}
        \PY{n}{mean\PYZus{}of\PYZus{}x1} \PY{o}{=} \PY{p}{(}\PY{n}{np}\PY{o}{.}\PY{n}{sum}\PY{p}{(}\PY{n}{x1}\PY{p}{)}\PY{o}{/}\PY{n}{x1}\PY{o}{.}\PY{n}{size}\PY{p}{)}
        \PY{n}{mean\PYZus{}of\PYZus{}x2} \PY{o}{=} \PY{p}{(}\PY{n}{np}\PY{o}{.}\PY{n}{sum}\PY{p}{(}\PY{n}{x2}\PY{p}{)}\PY{o}{/}\PY{n}{x2}\PY{o}{.}\PY{n}{size}\PY{p}{)}
        
        \PY{c+c1}{\PYZsh{}Mean of bivariate gaussian distribution is simply the sum of individual means of x1 and x2}
        \PY{n}{mean\PYZus{}of\PYZus{}bivariate\PYZus{}gaussian} \PY{o}{=} \PY{n}{mean\PYZus{}of\PYZus{}x1} \PY{o}{+} \PY{n}{mean\PYZus{}of\PYZus{}x2}
        \PY{n+nb}{print}\PY{p}{(}\PY{l+s+s2}{\PYZdq{}}\PY{l+s+s2}{Calculated mean of generated bivariate distribution is }\PY{l+s+si}{\PYZob{}\PYZcb{}}\PY{l+s+s2}{\PYZdq{}}\PY{o}{.}\PY{n}{format}\PY{p}{(}\PY{n}{mean\PYZus{}of\PYZus{}bivariate\PYZus{}gaussian}\PY{p}{)}\PY{p}{)}
        
        \PY{c+c1}{\PYZsh{}Calculating the variance of x1 to be used in covariance matrix}
        \PY{n}{distribution\PYZus{}subtracted\PYZus{}mean\PYZus{}square\PYZus{}of\PYZus{}x1} \PY{o}{=} \PY{p}{(}\PY{n}{x1} \PY{o}{\PYZhy{}} \PY{n}{mean\PYZus{}of\PYZus{}x1}\PY{p}{)}\PY{o}{*}\PY{p}{(}\PY{n}{x1}\PY{o}{\PYZhy{}}\PY{n}{mean\PYZus{}of\PYZus{}x1}\PY{p}{)}
        \PY{n}{variance\PYZus{}of\PYZus{}distribution\PYZus{}x1} \PY{o}{=} \PY{n}{np}\PY{o}{.}\PY{n}{sum}\PY{p}{(}\PY{n}{distribution\PYZus{}subtracted\PYZus{}mean\PYZus{}square\PYZus{}of\PYZus{}x1}\PY{p}{)}\PY{o}{/}\PY{n}{distribution\PYZus{}subtracted\PYZus{}mean\PYZus{}square\PYZus{}of\PYZus{}x1}\PY{o}{.}\PY{n}{size}
        
        \PY{c+c1}{\PYZsh{}Calculating the variance of x2 to be used in covariance matrix}
        \PY{n}{distribution\PYZus{}subtracted\PYZus{}mean\PYZus{}square\PYZus{}of\PYZus{}x2} \PY{o}{=} \PY{p}{(}\PY{n}{x2} \PY{o}{\PYZhy{}} \PY{n}{mean\PYZus{}of\PYZus{}x2}\PY{p}{)}\PY{o}{*}\PY{p}{(}\PY{n}{x2}\PY{o}{\PYZhy{}}\PY{n}{mean\PYZus{}of\PYZus{}x2}\PY{p}{)}
        \PY{n}{variance\PYZus{}of\PYZus{}distribution\PYZus{}x2} \PY{o}{=} \PY{n}{np}\PY{o}{.}\PY{n}{sum}\PY{p}{(}\PY{n}{distribution\PYZus{}subtracted\PYZus{}mean\PYZus{}square\PYZus{}of\PYZus{}x2}\PY{p}{)}\PY{o}{/}\PY{n}{distribution\PYZus{}subtracted\PYZus{}mean\PYZus{}square\PYZus{}of\PYZus{}x2}\PY{o}{.}\PY{n}{size}
        
        \PY{c+c1}{\PYZsh{}Calculating the covariance of x1 and x2 as expected value of (x1\PYZhy{}E[x1])*(x2\PYZhy{}E[x2])}
        \PY{n}{distribution\PYZus{}of\PYZus{}x1\PYZus{}and\PYZus{}x2\PYZus{}with\PYZus{}subtracted\PYZus{}mean} \PY{o}{=} \PY{p}{(}\PY{n}{x1}\PY{o}{\PYZhy{}}\PY{n}{mean\PYZus{}of\PYZus{}x1}\PY{p}{)}\PY{o}{*}\PY{p}{(}\PY{n}{x2}\PY{o}{\PYZhy{}}\PY{n}{mean\PYZus{}of\PYZus{}x2}\PY{p}{)}
        \PY{n}{covariance\PYZus{}of\PYZus{}x1\PYZus{}and\PYZus{}x2} \PY{o}{=} \PY{n}{np}\PY{o}{.}\PY{n}{sum}\PY{p}{(}\PY{n}{distribution\PYZus{}of\PYZus{}x1\PYZus{}and\PYZus{}x2\PYZus{}with\PYZus{}subtracted\PYZus{}mean}\PY{p}{)}\PY{o}{/}\PY{n}{distribution\PYZus{}of\PYZus{}x1\PYZus{}and\PYZus{}x2\PYZus{}with\PYZus{}subtracted\PYZus{}mean}\PY{o}{.}\PY{n}{size}
        
        \PY{c+c1}{\PYZsh{}Generating covariance matrix as list of lists}
        \PY{n}{covariance\PYZus{}matrix} \PY{o}{=} \PY{p}{[}\PY{p}{]}
        
        \PY{n}{covariance\PYZus{}matrix\PYZus{}row1} \PY{o}{=}\PY{p}{[}\PY{n}{variance\PYZus{}of\PYZus{}distribution\PYZus{}x1}\PY{p}{,}\PY{n}{covariance\PYZus{}of\PYZus{}x1\PYZus{}and\PYZus{}x2}\PY{p}{]}
        \PY{n}{covariance\PYZus{}matrix}\PY{o}{.}\PY{n}{append}\PY{p}{(}\PY{n}{covariance\PYZus{}matrix\PYZus{}row1}\PY{p}{)}
        
        \PY{n}{covariance\PYZus{}matrix\PYZus{}row2} \PY{o}{=} \PY{p}{[}\PY{n}{covariance\PYZus{}of\PYZus{}x1\PYZus{}and\PYZus{}x2}\PY{p}{,}\PY{n}{variance\PYZus{}of\PYZus{}distribution\PYZus{}x2}\PY{p}{]}
        \PY{n}{covariance\PYZus{}matrix}\PY{o}{.}\PY{n}{append}\PY{p}{(}\PY{n}{covariance\PYZus{}matrix\PYZus{}row2}\PY{p}{)}
        
        \PY{c+c1}{\PYZsh{}converting covariance matrix into data frame for better visual display}
        \PY{n}{data\PYZus{}frame} \PY{o}{=} \PY{n}{pd}\PY{o}{.}\PY{n}{DataFrame}\PY{p}{(}\PY{n}{covariance\PYZus{}matrix}\PY{p}{)}
        
        \PY{c+c1}{\PYZsh{}changing index from numbering to distribtution names}
        
        \PY{n}{data\PYZus{}frame}\PY{o}{.}\PY{n}{rename}\PY{p}{(}\PY{n}{index}\PY{o}{=}\PY{p}{\PYZob{}}\PY{l+m+mi}{0}\PY{p}{:}\PY{l+s+s1}{\PYZsq{}}\PY{l+s+s1}{x1}\PY{l+s+s1}{\PYZsq{}}\PY{p}{,}\PY{l+m+mi}{1}\PY{p}{:}\PY{l+s+s1}{\PYZsq{}}\PY{l+s+s1}{x2}\PY{l+s+s1}{\PYZsq{}}\PY{p}{\PYZcb{}}\PY{p}{,} \PY{n}{columns}\PY{o}{=}\PY{p}{\PYZob{}}\PY{l+m+mi}{0}\PY{p}{:}\PY{l+s+s1}{\PYZsq{}}\PY{l+s+s1}{x1}\PY{l+s+s1}{\PYZsq{}}\PY{p}{,}\PY{l+m+mi}{1}\PY{p}{:}\PY{l+s+s1}{\PYZsq{}}\PY{l+s+s1}{x2}\PY{l+s+s1}{\PYZsq{}}\PY{p}{\PYZcb{}}\PY{p}{,} \PY{n}{inplace}\PY{o}{=}\PY{k+kc}{True}\PY{p}{)}
        \PY{n+nb}{print}\PY{p}{(}\PY{l+s+s2}{\PYZdq{}}\PY{l+s+s2}{Covariance matrix of the generated bivariate gaussian distribution is displayed below}\PY{l+s+s2}{\PYZdq{}}\PY{p}{)}
        \PY{n+nb}{print}\PY{p}{(}\PY{n}{data\PYZus{}frame}\PY{p}{)}
\end{Verbatim}


    \begin{Verbatim}[commandchars=\\\{\}]
Calculated mean of generated bivariate distribution is -0.047783869093271925
Covariance matrix of the generated bivariate gaussian distribution is displayed below
           x1         x2
x1  19.683774   0.621428
x2   0.621428  30.195189

    \end{Verbatim}

    \section{Problem 5}\label{problem-5}

Download from Canvas/Files the dataset PatientData.csv. Each row is a
patient and the last column is the condition that the patient has. Do
data exploration using Pandas and other visualization tools to
understand what you can about the dataset.

For example: 1. How many patients and how many features are there? 2.
What is the meaning of the first 4 features? See if you can understand
what they mean. 3. Are there missing values? Replace them with the
average of the corresponding feature column. 4. How could you test which
features strongly influence the patient condition and which do not?

List what you think are the three most important features.

    \begin{Verbatim}[commandchars=\\\{\}]
{\color{incolor}In [{\color{incolor}7}]:} \PY{k+kn}{import} \PY{n+nn}{pandas} \PY{k}{as} \PY{n+nn}{pd}
        \PY{k+kn}{import} \PY{n+nn}{matplotlib}\PY{n+nn}{.}\PY{n+nn}{pyplot} \PY{k}{as} \PY{n+nn}{plt}
        
        \PY{n}{file\PYZus{}path} \PY{o}{=} \PY{l+s+s1}{\PYZsq{}}\PY{l+s+s1}{PatientData.csv}\PY{l+s+s1}{\PYZsq{}}
        \PY{n}{patient\PYZus{}data} \PY{o}{=} \PY{n}{pd}\PY{o}{.}\PY{n}{read\PYZus{}csv}\PY{p}{(}\PY{n}{file\PYZus{}path}\PY{p}{)}
        \PY{n}{patient\PYZus{}data}\PY{o}{.}\PY{n}{columns}
        \PY{n}{patient\PYZus{}data}\PY{o}{.}\PY{n}{values}
\end{Verbatim}


\begin{Verbatim}[commandchars=\\\{\}]
{\color{outcolor}Out[{\color{outcolor}7}]:} array([[56, 1, 165, {\ldots}, 20.4, 38.8, 6],
               [54, 0, 172, {\ldots}, 12.3, 49.0, 10],
               [55, 0, 175, {\ldots}, 34.6, 61.6, 1],
               {\ldots},
               [36, 0, 166, {\ldots}, -44.2, -33.2, 2],
               [32, 1, 155, {\ldots}, 25.0, 46.6, 1],
               [78, 1, 160, {\ldots}, 21.3, 32.8, 1]], dtype=object)
\end{Verbatim}
            
    \subsection{Part 1}\label{part-1}

Calling \texttt{patient\_data.shape} shows the patients and features.
There is one less feature than the dimension along the second axis, as
the final vector is used for the labels.

    \begin{Verbatim}[commandchars=\\\{\}]
{\color{incolor}In [{\color{incolor}8}]:} \PY{n}{patient\PYZus{}data}\PY{o}{.}\PY{n}{shape}
        \PY{c+c1}{\PYZsh{} 451 patients}
        \PY{c+c1}{\PYZsh{} 279 features}
\end{Verbatim}


\begin{Verbatim}[commandchars=\\\{\}]
{\color{outcolor}Out[{\color{outcolor}8}]:} (451, 280)
\end{Verbatim}
            
    \subsection{Part 2}\label{part-2}

Here we pull the first four columns of our data set, and plot them.

    \begin{Verbatim}[commandchars=\\\{\}]
{\color{incolor}In [{\color{incolor}9}]:} \PY{n}{first\PYZus{}four} \PY{o}{=} \PY{p}{[}\PY{p}{]}
        \PY{n}{i} \PY{o}{=} \PY{l+m+mi}{0}
        \PY{k}{for} \PY{n}{i} \PY{o+ow}{in} \PY{n+nb}{range}\PY{p}{(}\PY{l+m+mi}{4}\PY{p}{)}\PY{p}{:}
            \PY{n}{first\PYZus{}four}\PY{o}{.}\PY{n}{append}\PY{p}{(}\PY{n}{patient\PYZus{}data}\PY{o}{.}\PY{n}{iloc}\PY{p}{[}\PY{p}{:}\PY{p}{,}\PY{n}{i}\PY{p}{]}\PY{p}{)}
        
        \PY{k}{for} \PY{n}{i} \PY{o+ow}{in} \PY{n+nb}{range}\PY{p}{(}\PY{l+m+mi}{4}\PY{p}{)}\PY{p}{:}
            \PY{n}{plt}\PY{o}{.}\PY{n}{hist}\PY{p}{(}\PY{n}{first\PYZus{}four}\PY{p}{[}\PY{n}{i}\PY{p}{]}\PY{p}{,} \PY{n}{bins} \PY{o}{=} \PY{l+m+mi}{20}\PY{p}{)}
            \PY{n}{plt}\PY{o}{.}\PY{n}{title}\PY{p}{(}\PY{l+s+s1}{\PYZsq{}}\PY{l+s+s1}{Column }\PY{l+s+si}{\PYZob{}\PYZcb{}}\PY{l+s+s1}{\PYZsq{}}\PY{o}{.}\PY{n}{format}\PY{p}{(}\PY{n}{i}\PY{p}{)}\PY{p}{)}
            \PY{n}{plt}\PY{o}{.}\PY{n}{show}\PY{p}{(}\PY{p}{)}
\end{Verbatim}


    \begin{center}
    \adjustimage{max size={0.9\linewidth}{0.9\paperheight}}{output_15_0.png}
    \end{center}
    { \hspace*{\fill} \\}
    
    \begin{center}
    \adjustimage{max size={0.9\linewidth}{0.9\paperheight}}{output_15_1.png}
    \end{center}
    { \hspace*{\fill} \\}
    
    \begin{center}
    \adjustimage{max size={0.9\linewidth}{0.9\paperheight}}{output_15_2.png}
    \end{center}
    { \hspace*{\fill} \\}
    
    \begin{center}
    \adjustimage{max size={0.9\linewidth}{0.9\paperheight}}{output_15_3.png}
    \end{center}
    { \hspace*{\fill} \\}
    
    We plot scatters here to get a better idea of the data representation

    \begin{Verbatim}[commandchars=\\\{\}]
{\color{incolor}In [{\color{incolor}11}]:} \PY{k+kn}{import} \PY{n+nn}{numpy} \PY{k}{as} \PY{n+nn}{np}
         \PY{n}{x} \PY{o}{=} \PY{n}{np}\PY{o}{.}\PY{n}{arange}\PY{p}{(}\PY{n+nb}{len}\PY{p}{(}\PY{n}{first\PYZus{}four}\PY{p}{[}\PY{l+m+mi}{1}\PY{p}{]}\PY{p}{)}\PY{p}{)}
         \PY{n}{plt}\PY{o}{.}\PY{n}{figure}\PY{p}{(}\PY{n}{figsize}\PY{o}{=}\PY{p}{(}\PY{l+m+mi}{10}\PY{p}{,}\PY{l+m+mi}{6}\PY{p}{)}\PY{p}{)}
         \PY{n}{plt}\PY{o}{.}\PY{n}{scatter}\PY{p}{(}\PY{n}{x}\PY{p}{,} \PY{n}{first\PYZus{}four}\PY{p}{[}\PY{l+m+mi}{1}\PY{p}{]}\PY{p}{,} \PY{n}{s}\PY{o}{=}\PY{l+m+mi}{3}\PY{p}{,} \PY{n}{alpha}\PY{o}{=}\PY{l+m+mf}{0.7}\PY{p}{)}
         \PY{n}{plt}\PY{o}{.}\PY{n}{xlabel}\PY{p}{(}\PY{l+s+s1}{\PYZsq{}}\PY{l+s+s1}{Patients}\PY{l+s+s1}{\PYZsq{}}\PY{p}{,} \PY{n}{fontsize} \PY{o}{=} \PY{l+s+s1}{\PYZsq{}}\PY{l+s+s1}{25}\PY{l+s+s1}{\PYZsq{}}\PY{p}{)}
         \PY{n}{plt}\PY{o}{.}\PY{n}{ylabel}\PY{p}{(}\PY{l+s+s1}{\PYZsq{}}\PY{l+s+s1}{Unknown Classification}\PY{l+s+s1}{\PYZsq{}}\PY{p}{,} \PY{n}{fontsize} \PY{o}{=} \PY{l+s+s1}{\PYZsq{}}\PY{l+s+s1}{25}\PY{l+s+s1}{\PYZsq{}}\PY{p}{)}
         \PY{n}{plt}\PY{o}{.}\PY{n}{title}\PY{p}{(}\PY{l+s+s1}{\PYZsq{}}\PY{l+s+s1}{Scatter of Unknown Feature, Column 1}\PY{l+s+s1}{\PYZsq{}}\PY{p}{,} \PY{n}{fontsize}\PY{o}{=}\PY{l+s+s1}{\PYZsq{}}\PY{l+s+s1}{35}\PY{l+s+s1}{\PYZsq{}}\PY{p}{)}
         \PY{n}{plt}\PY{o}{.}\PY{n}{show}\PY{p}{(}\PY{p}{)}
\end{Verbatim}


    \begin{center}
    \adjustimage{max size={0.9\linewidth}{0.9\paperheight}}{output_17_0.png}
    \end{center}
    { \hspace*{\fill} \\}
    
    \begin{Verbatim}[commandchars=\\\{\}]
{\color{incolor}In [{\color{incolor}14}]:} \PY{k+kn}{import} \PY{n+nn}{numpy} \PY{k}{as} \PY{n+nn}{np}
         \PY{n}{x} \PY{o}{=} \PY{n}{np}\PY{o}{.}\PY{n}{arange}\PY{p}{(}\PY{n+nb}{len}\PY{p}{(}\PY{n}{first\PYZus{}four}\PY{p}{[}\PY{l+m+mi}{2}\PY{p}{]}\PY{p}{)}\PY{p}{)}
         \PY{n}{fig} \PY{o}{=} \PY{n}{plt}\PY{o}{.}\PY{n}{figure}\PY{p}{(}\PY{n}{figsize}\PY{o}{=}\PY{p}{(}\PY{l+m+mi}{10}\PY{p}{,}\PY{l+m+mi}{6}\PY{p}{)}\PY{p}{)}
         \PY{n}{ax} \PY{o}{=} \PY{n}{fig}\PY{o}{.}\PY{n}{gca}\PY{p}{(}\PY{p}{)}
         \PY{n}{ax}\PY{o}{.}\PY{n}{grid}\PY{p}{(}\PY{n}{linestyle}\PY{o}{=}\PY{l+s+s1}{\PYZsq{}}\PY{l+s+s1}{\PYZhy{}}\PY{l+s+s1}{\PYZsq{}}\PY{p}{,} \PY{n}{linewidth}\PY{o}{=}\PY{l+m+mf}{0.7}\PY{p}{)}
         \PY{n}{plt}\PY{o}{.}\PY{n}{scatter}\PY{p}{(}\PY{n}{x}\PY{p}{,} \PY{n}{first\PYZus{}four}\PY{p}{[}\PY{l+m+mi}{2}\PY{p}{]}\PY{p}{,} \PY{n}{s}\PY{o}{=}\PY{l+m+mi}{25}\PY{p}{,} \PY{n}{alpha}\PY{o}{=}\PY{l+m+mi}{1}\PY{p}{,} \PY{n}{marker}\PY{o}{=}\PY{l+s+s1}{\PYZsq{}}\PY{l+s+s1}{*}\PY{l+s+s1}{\PYZsq{}}\PY{p}{)}
         \PY{n}{plt}\PY{o}{.}\PY{n}{xlabel}\PY{p}{(}\PY{l+s+s1}{\PYZsq{}}\PY{l+s+s1}{Patients}\PY{l+s+s1}{\PYZsq{}}\PY{p}{,} \PY{n}{fontsize} \PY{o}{=} \PY{l+s+s1}{\PYZsq{}}\PY{l+s+s1}{25}\PY{l+s+s1}{\PYZsq{}}\PY{p}{)}
         \PY{n}{plt}\PY{o}{.}\PY{n}{ylim}\PY{p}{(}\PY{p}{[}\PY{l+m+mi}{100}\PY{p}{,}\PY{l+m+mi}{200}\PY{p}{]}\PY{p}{)}
         \PY{n}{plt}\PY{o}{.}\PY{n}{yticks}\PY{p}{(}\PY{n}{np}\PY{o}{.}\PY{n}{arange}\PY{p}{(}\PY{l+m+mi}{100}\PY{p}{,} \PY{l+m+mi}{200}\PY{p}{,} \PY{l+m+mf}{5.0}\PY{p}{)}\PY{p}{)}
         \PY{n}{plt}\PY{o}{.}\PY{n}{xticks}\PY{p}{(}\PY{n}{np}\PY{o}{.}\PY{n}{arange}\PY{p}{(}\PY{l+m+mi}{0}\PY{p}{,} \PY{l+m+mi}{451}\PY{p}{,} \PY{l+m+mf}{20.0}\PY{p}{)}\PY{p}{)}
         \PY{n}{plt}\PY{o}{.}\PY{n}{ylabel}\PY{p}{(}\PY{l+s+s1}{\PYZsq{}}\PY{l+s+s1}{Unknown Classification}\PY{l+s+s1}{\PYZsq{}}\PY{p}{,} \PY{n}{fontsize} \PY{o}{=} \PY{l+s+s1}{\PYZsq{}}\PY{l+s+s1}{25}\PY{l+s+s1}{\PYZsq{}}\PY{p}{)}
         \PY{n}{plt}\PY{o}{.}\PY{n}{title}\PY{p}{(}\PY{l+s+s1}{\PYZsq{}}\PY{l+s+s1}{Scatter of Unknown Feature, Column 2}\PY{l+s+s1}{\PYZsq{}}\PY{p}{,} \PY{n}{fontsize}\PY{o}{=}\PY{l+s+s1}{\PYZsq{}}\PY{l+s+s1}{35}\PY{l+s+s1}{\PYZsq{}}\PY{p}{)}
         \PY{n}{plt}\PY{o}{.}\PY{n}{show}\PY{p}{(}\PY{p}{)}
\end{Verbatim}


    \begin{center}
    \adjustimage{max size={0.9\linewidth}{0.9\paperheight}}{output_18_0.png}
    \end{center}
    { \hspace*{\fill} \\}
    
    \begin{Verbatim}[commandchars=\\\{\}]
{\color{incolor}In [{\color{incolor}15}]:} \PY{k+kn}{from} \PY{n+nn}{pandas}\PY{n+nn}{.}\PY{n+nn}{plotting} \PY{k}{import} \PY{n}{scatter\PYZus{}matrix}
         \PY{n}{columns} \PY{o}{=} \PY{l+m+mi}{4}
         \PY{n}{scatter\PYZus{}matrix}\PY{p}{(}\PY{n}{patient\PYZus{}data}\PY{o}{.}\PY{n}{iloc}\PY{p}{[}\PY{p}{:}\PY{p}{,}\PY{p}{:}\PY{l+m+mi}{4}\PY{p}{]}\PY{p}{,} \PY{n}{figsize}\PY{o}{=}\PY{p}{(}\PY{l+m+mi}{12}\PY{p}{,}\PY{l+m+mi}{8}\PY{p}{)}\PY{p}{)}
         \PY{n}{plt}\PY{o}{.}\PY{n}{show}
\end{Verbatim}


\begin{Verbatim}[commandchars=\\\{\}]
{\color{outcolor}Out[{\color{outcolor}15}]:} <function matplotlib.pyplot.show>
\end{Verbatim}
            
    \begin{center}
    \adjustimage{max size={0.9\linewidth}{0.9\paperheight}}{output_19_1.png}
    \end{center}
    { \hspace*{\fill} \\}
    
    \begin{Verbatim}[commandchars=\\\{\}]
{\color{incolor}In [{\color{incolor}16}]:} \PY{n}{df} \PY{o}{=} \PY{n}{pd}\PY{o}{.}\PY{n}{concat}\PY{p}{(}\PY{p}{[}\PY{n}{patient\PYZus{}data}\PY{p}{[}\PY{l+s+s1}{\PYZsq{}}\PY{l+s+s1}{190}\PY{l+s+s1}{\PYZsq{}}\PY{p}{]}\PY{p}{,} \PY{n}{patient\PYZus{}data}\PY{p}{[}\PY{l+s+s1}{\PYZsq{}}\PY{l+s+s1}{8}\PY{l+s+s1}{\PYZsq{}}\PY{p}{]}\PY{p}{]}\PY{p}{,} \PY{n}{axis}\PY{o}{=}\PY{l+m+mi}{1}\PY{p}{,} \PY{n}{keys}\PY{o}{=}\PY{p}{[}\PY{l+s+s1}{\PYZsq{}}\PY{l+s+s1}{Weight}\PY{l+s+s1}{\PYZsq{}}\PY{p}{,} \PY{l+s+s1}{\PYZsq{}}\PY{l+s+s1}{Labels}\PY{l+s+s1}{\PYZsq{}}\PY{p}{]}\PY{p}{)}
         \PY{n}{scatter\PYZus{}matrix}\PY{p}{(}\PY{n}{df}\PY{p}{,} \PY{n}{figsize}\PY{o}{=}\PY{p}{(}\PY{l+m+mi}{12}\PY{p}{,}\PY{l+m+mi}{8}\PY{p}{)}\PY{p}{)}
\end{Verbatim}


\begin{Verbatim}[commandchars=\\\{\}]
{\color{outcolor}Out[{\color{outcolor}16}]:} array([[<matplotlib.axes.\_subplots.AxesSubplot object at 0x7fc42d1d8940>,
                 <matplotlib.axes.\_subplots.AxesSubplot object at 0x7fc42d252e48>],
                [<matplotlib.axes.\_subplots.AxesSubplot object at 0x7fc42d0199b0>,
                 <matplotlib.axes.\_subplots.AxesSubplot object at 0x7fc42d02ae48>]],
               dtype=object)
\end{Verbatim}
            
    \begin{center}
    \adjustimage{max size={0.9\linewidth}{0.9\paperheight}}{output_20_1.png}
    \end{center}
    { \hspace*{\fill} \\}
    
    \subsection{Part 2 Conclusion}\label{part-2-conclusion}

In the end, we conclude the columns are as follows: 0. Age

Since the data in column zero goes from to just past 80, we felt that
age was a good guess for the data in this column. There were no outliers
(nobody that is 200 years old), no negative values, and what looks like
something close to a normal distribution of data.

\begin{enumerate}
\def\labelenumi{\arabic{enumi}.}
\tightlist
\item
  Gender

  This guess is mostly based on the binary nature of the data, and the
  fact that it is pretty evenly distributed between 0 and 1. We could
  not think of another important evenly distributed binary data point.
\item
  Height in centimeters

  This column took a little bit longer to figure out, we had to plot it
  in a few different ways, before finally realizing what it might be.
  The clue was realizing, what do most people have that is between 160
  and 180? Our first thought was weight in lbs., but we realized this
  wasn't quite right. Thinking that data may be metric, and that may be
  why we don't recognize it, a quick check of 160 cm -\textgreater{}
  feet and 180 cm -\textgreater{} feet confirmed our suspicions.
\item
  Income

  The only good reasoning on this is just looking at data for average
  salaries. Most people make between 40-80 thousand per year, with the
  average in Austin specifically being around \$55,000.
\end{enumerate}

    \subsection{Part 3}\label{part-3}

We calculate the mean of each column, and pass it to the pandas command
to fill NaN values.

    \begin{Verbatim}[commandchars=\\\{\}]
{\color{incolor}In [{\color{incolor}20}]:} \PY{n}{patient\PYZus{}data}\PY{o}{.}\PY{n}{fillna}\PY{p}{(}\PY{n}{patient\PYZus{}data}\PY{o}{.}\PY{n}{mean}\PY{p}{(}\PY{p}{)}\PY{p}{)}
         \PY{c+c1}{\PYZsh{}assert(patient\PYZus{}data.isnull().values.any() == False), \PYZdq{}Try again\PYZdq{}}
         \PY{k}{if} \PY{n}{patient\PYZus{}data}\PY{o}{.}\PY{n}{isnull}\PY{p}{(}\PY{p}{)}\PY{o}{.}\PY{n}{values}\PY{o}{.}\PY{n}{any}\PY{p}{(}\PY{p}{)} \PY{o}{==} \PY{k+kc}{False}\PY{p}{:}
             \PY{n+nb}{print}\PY{p}{(}\PY{l+s+s2}{\PYZdq{}}\PY{l+s+s2}{Success!}\PY{l+s+s2}{\PYZdq{}}\PY{p}{)}
         \PY{k}{else}\PY{p}{:}
             \PY{n+nb}{print}\PY{p}{(}\PY{l+s+s2}{\PYZdq{}}\PY{l+s+s2}{Failure!}\PY{l+s+s2}{\PYZdq{}}\PY{p}{)}
\end{Verbatim}


    \begin{Verbatim}[commandchars=\\\{\}]
Success!

    \end{Verbatim}

    \subsection{Part 4}\label{part-4}

There are various ways to map the data to the classification given.
Investigating the columns with higher correlation to the outcomes could
help pinpoint features that have more influence. For example, there are
built in methods that can produce a matrix showing correlations between
the columns. Checking which columns have greater (absolute value)
correlations with the label column would help show the features that
more strongly influence the labels.

A classifier such as naive bayes would likely work. Essentially you are
looking for a mapping from relevant features to labels.

    \subsection{Extra}\label{extra}

Various calls that we used to try to look at the data at different
points

    \begin{Verbatim}[commandchars=\\\{\}]
{\color{incolor}In [{\color{incolor}21}]:} \PY{n}{patient\PYZus{}data}\PY{o}{.}\PY{n}{describe}\PY{p}{(}\PY{p}{)}
\end{Verbatim}


\begin{Verbatim}[commandchars=\\\{\}]
{\color{outcolor}Out[{\color{outcolor}21}]:}                75           0         190          80          91         193  \textbackslash{}
         count  451.000000  451.000000  451.000000  451.000000  451.000000  451.000000   
         mean    46.407982    0.552106  166.135255   68.144124   88.915743  155.068736   
         std     16.429846    0.497830   37.194646   16.599841   15.381143   44.856534   
         min      0.000000    0.000000  105.000000    6.000000   55.000000    0.000000   
         25\%     36.000000    0.000000  160.000000   59.000000   80.000000  142.000000   
         50\%     47.000000    1.000000  164.000000   68.000000   86.000000  157.000000   
         75\%     58.000000    1.000000  170.000000   78.500000   94.000000  174.500000   
         max     83.000000    1.000000  780.000000  176.000000  188.000000  524.000000   
         
                       371         174         121         -16     {\ldots}          0.0.38  \textbackslash{}
         count  451.000000  451.000000  451.000000  451.000000     {\ldots}      451.000000   
         mean   367.199557  169.940133   89.935698   33.787140     {\ldots}       -0.279601   
         std     33.422017   35.672130   25.813912   45.421423     {\ldots}        0.549328   
         min    232.000000  108.000000    0.000000 -172.000000     {\ldots}       -4.100000   
         25\%    350.000000  148.000000   79.000000    4.000000     {\ldots}       -0.450000   
         50\%    367.000000  162.000000   91.000000   40.000000     {\ldots}        0.000000   
         75\%    384.000000  179.000000  102.000000   66.000000     {\ldots}        0.000000   
         max    509.000000  381.000000  205.000000  169.000000     {\ldots}        0.000000   
         
                       9.0        -0.9      0.0.39  0.0.40       0.9.3       2.9.1  \textbackslash{}
         count  451.000000  451.000000  451.000000   451.0  451.000000  451.000000   
         mean     9.048115   -1.458537    0.003991     0.0    0.513969    1.218625   
         std      3.476718    2.004481    0.050173     0.0    0.347441    1.425438   
         min      0.000000  -28.600000    0.000000     0.0   -0.800000   -6.000000   
         25\%      6.600000   -2.100000    0.000000     0.0    0.400000    0.500000   
         50\%      8.800000   -1.100000    0.000000     0.0    0.500000    1.300000   
         75\%     11.200000    0.000000    0.000000     0.0    0.700000    2.100000   
         max     23.600000    0.000000    0.800000     0.0    2.400000    6.000000   
         
                      23.3        49.4           8  
         count  451.000000  451.000000  451.000000  
         mean    19.317295   29.429047    3.871397  
         std     13.517617   18.490566    4.407706  
         min    -44.200000  -38.600000    1.000000  
         25\%     11.400000   17.500000    1.000000  
         50\%     18.100000   27.900000    1.000000  
         75\%     25.850000   41.050000    6.000000  
         max     88.800000  115.900000   16.000000  
         
         [8 rows x 275 columns]
\end{Verbatim}
            
    \begin{Verbatim}[commandchars=\\\{\}]
{\color{incolor}In [{\color{incolor}22}]:} \PY{n}{patient\PYZus{}data}\PY{o}{.}\PY{n}{info}\PY{p}{(}\PY{p}{)}
\end{Verbatim}


    \begin{Verbatim}[commandchars=\\\{\}]
<class 'pandas.core.frame.DataFrame'>
RangeIndex: 451 entries, 0 to 450
Columns: 280 entries, 75 to 8
dtypes: float64(120), int64(155), object(5)
memory usage: 986.6+ KB

    \end{Verbatim}

    \begin{Verbatim}[commandchars=\\\{\}]
{\color{incolor}In [{\color{incolor}17}]:} \PY{n}{patient\PYZus{}data}\PY{o}{.}\PY{n}{head}\PY{p}{(}\PY{p}{)}
\end{Verbatim}


\begin{Verbatim}[commandchars=\\\{\}]
{\color{outcolor}Out[{\color{outcolor}17}]:}    75  0  190  80   91  193  371  174  121  -16 {\ldots} 0.0.38   9.0 -0.9 0.0.39  \textbackslash{}
         0  56  1  165  64   81  174  401  149   39   25 {\ldots}    0.0   8.5  0.0    0.0   
         1  54  0  172  95  138  163  386  185  102   96 {\ldots}    0.0   9.5 -2.4    0.0   
         2  55  0  175  94  100  202  380  179  143   28 {\ldots}    0.0  12.2 -2.2    0.0   
         3  75  0  190  80   88  181  360  177  103  -16 {\ldots}    0.0  13.1 -3.6    0.0   
         4  13  0  169  51  100  167  321  174   91  107 {\ldots}   -0.6  12.2 -2.8    0.0   
         
           0.0.40  0.9.3  2.9.1  23.3  49.4   8  
         0    0.0    0.2    2.1  20.4  38.8   6  
         1    0.0    0.3    3.4  12.3  49.0  10  
         2    0.0    0.4    2.6  34.6  61.6   1  
         3    0.0   -0.1    3.9  25.4  62.8   7  
         4    0.0    0.9    2.2  13.5  31.1  14  
         
         [5 rows x 280 columns]
\end{Verbatim}
            

    % Add a bibliography block to the postdoc
    
    
    
    \end{document}
